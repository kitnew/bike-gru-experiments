% !TeX root=tukedip.tex
% !TeX encoding = UTF-8
% !TeX spellcheck = sk_SK
%\section{Analýza}

\section{Analýza existujúcich riešení a metód}

V tejto kapitole sa zaoberáme analýzou existujúcich prístupov k optimalizácii hyperparametrov rekurentných neurónových sietí s dôrazom na architektúru GRU. Podrobne skúmame hlavné metódy optimalizácie hyperparametrov a analyzujeme ich výhody a nevýhody v kontexte použitia na rôznych dátových množinách.

\subsection{Metódy optimalizácie hyperparametrov}

Optimálne nastavenie hyperparametrov je kľúčové pre dosiahnutie vysokej presnosti modelov strojového učenia. Existujú rôzne prístupy na riešenie tejto úlohy:

\textbf{Grid Search} predstavuje systematické skúmanie vopred definovanej množiny hyperparametrov, kde každý parameter je testovaný v rôznych kombináciách s ostatnými. Táto metóda je jednoduchá na implementáciu, ale neefektívna pri veľkom počte hyperparametrov, keďže počet potrebných výpočtov rastie exponenciálne.

\textbf{Random Search} vykonáva náhodný výber kombinácií hyperparametrov z definovaného rozsahu. Táto metóda je efektívnejšia ako grid search v prípadoch s veľkým počtom parametrov, pretože poskytuje širšie pokrytie priestoru hyperparametrov s menším počtom iterácií.

\textbf{Bayesian Search} využíva Bayesovské metódy na riadenie výberu kombinácií hyperparametrov. Táto metóda postupne zlepšuje svoj výber na základe predchádzajúcich výsledkov a je efektívna najmä pri zložitých modeloch, kde tradičné metódy optimalizácie môžu byť neefektívne.

Postup optimalizácie bol nasledovný: najprv bol vykonaný random a Bayesian search na rovnakom rozsahu hyperparametrov. Následne boli tri najlepšie sady hyperparametrov z oboch metód kombinované a pomocou grid search bola nájdená najlepšia výsledná kombinácia. Tento prístup umožnil efektívne zúženie priestoru hľadania a zvýšenie presnosti finálneho výberu hyperparametrov.

\subsection{Analýza dostupných metrík hodnotenia}

Pre vyhodnotenie výkonu modelov strojového učenia sa používajú rôzne metriky. V rámci tejto práce sme manuálne implementovali niekoľko štandardných metrík (\ref{sec:metrics}) s využitím knižníc numpy a pandas:

\begin{itemize}
\item \textbf{MSE (Mean Squared Error)}: Priemerná štvorcová chyba meria priemernú veľkosť chyby medzi predikovanými a reálnymi hodnotami.
\item \textbf{MAE (Mean Absolute Error)}: Priemerná absolútna chyba, robustnejšia metrika voči odľahlým hodnotám.
\item \textbf{RMSE (Root Mean Squared Error)}: Odmocnina z MSE, poskytuje chybu v jednotkách pôvodnej veličiny, čím zlepšuje interpretovateľnosť výsledkov.
\item \textbf{R-squared (koeficient determinácie)}: Udáva mieru zhody medzi modelom a dátami, pričom vyššie hodnoty naznačujú lepšiu predikčnú schopnosť.
\item \textbf{MAPE (Mean Absolute Percentage Error)}: Priemerná absolútna percentuálna chyba, vhodná na vyhodnotenie relatívnej veľkosti chyby.
\item \textbf{sMAPE (Symmetric Mean Absolute Percentage Error)}: Symetrická verzia MAPE, vhodná pre symetrické posúdenie chyby.
\item \textbf{Explained Variance}: Vyjadruje podiel vysvetlenej variancie modelom oproti celkovej variancii dát.
\item \textbf{Peak Error}: Udáva najväčšiu chybu medzi predikciou a reálnymi hodnotami, dôležitá pre analýzu kritických chýb.
\end{itemize}

Tieto metriky poskytujú jasný pohľad na presnosť a efektivitu vyvinutého algoritmu v rámci tejto práce. Pri použití na reálnych dátach poskytujú objektívne a konzistentné hodnotenie výkonu modelu.

\subsection{Analýza realizácií GRU modelov}

Architektúra GRU (\ref{sec:gru_model}) môže byť implementovaná pomocou rôznych knižníc strojového učenia, najpopulárnejšie sú TensorFlow a PyTorch. TensorFlow poskytuje širokú podporu, veľkú komunitu a množstvo integrovaných nástrojov, avšak PyTorch je častejšie preferovaný pre výskumné účely vďaka svojej jednoduchosti, flexibilite a dynamickému výpočtovému grafu.

Pre implementáciu bol zvolený práve PyTorch, pretože umožňuje jednoduché ladenie, poskytuje lepšiu transparentnosť pri experimentovaní a jednoduchšie nasadenie modelov v prostredí Jupyter notebook, ktoré bolo použité na interaktívnu analýzu a vizualizáciu výsledkov experimentov. Výber PyTorch bol strategickým rozhodnutím s cieľom maximalizovať produktivitu a efektivitu procesu vývoja.

% \subsection{podnapis}
% Text záverečnej práce obsahuje kapitolu, v~rámci ktorej autor uvedie
% analýzu riešených problémov. Táto kapitola môže byť v~prípade potreby
% delená do viacerých podkapitol. Autor v~texte záverečnej práce môže
% zvýrazniť kľúčové slová, pričom sa použije príslušný štýl pre kľúčové
% slová -- napr. toto je kľúčové slovo. V~texte môžu byť použité obrázky
% a~tabuľky podľa nasledujúcich príkladov:
% 
% \begin{figure}[!ht]
% \centering \unitlength=1mm
% \begin{picture}(30,30)(0,0)
% \put(0,0){\line(1,0){30}}
% \put(0,0){\line(0,1){30}}
% \put(30,0){\line(0,1){30}}
% \put(0,30){\line(1,0){30}}
% \end{picture}
% \caption{Toto je štvorec}\label{o:1}
% \end{figure}
% 
% 
% Obrázok by mal byť podľa možnosti centrovaný. Pri jeho opisovaní
% v~texte treba použiť odkazy na obrázok v~tvare Obrázok~\ref{o:1}.
% 
% %\tabcolsep=8pt
% \begin{table}[!ht]\caption{Prehľad jednotiek}\label{t:1}
% \smallskip
% \centering
% \begin{tabular}{|l|c|} \hline
% Názov	& (Jednotka v~sústave SI) \\ \hline
% Napätie & $\upmu$V \\ \hline
% \end{tabular}
% \end{table}
% \nomenclature{$\upmu$}{mikro, $10^{-6}$}
% \nomenclature{SI}{Syst\`eme International}
% \nomenclature{V}{volt, základná jednotka napätia v sústave SI}
% 
% Tabuľka by mala byť podľa možnosti centrovaná. Pri jej opisovaní
% v~texte treba použiť odkazy na tabuľku v~tvare: pozri
% Tabuľku~\ref{t:1}. Na číslovanie obrázkov, resp. tabuliek treba použiť
% triedenie. Za slovom {\it Obrázok} nasleduje ako prvé číslo kapitoly
% alebo časti, v~ktorej sa obrázok nachádza, potom medzera, pomlčka,
% medzera a~poradové číslo ilustrácie v~danej kapitole alebo časti.
% Napr.:~Obrázok~\ref{o:1} (čiže: prvý obrázok v~druhej kapitole alebo
% časti). V~prípade, ak tabuľka presahuje stranu, je možné použiť balík
% \verb+longtable+.
% 
% Navrhujeme zaraďovať obrázky v~elektronickej podobe. Napríklad
% Obrázok~\ref{o:2}, ktorý opisuje riešenie diferenciálnej rovnice
% tlmených oscilácií
% %% \def\ud{\mathrm{d}}
% 
% 
% \begin{equation}\label{r:1}
% \frac{\ud^2y}{\ud t^2}+\frac{\ud y}{\ud t}+y =0, \qquad y(0)=1, \quad
% y\,'(0)=15,
% \end{equation}
% 
% 
% 
% 
% bol vytvorený v~MATLABe a~príkazom \texttt{print tlmosc.eps -f1
% -deps2} bol uložený vo formáte Encapsulated Postscript. Na prípadné
% použitie pdf\LaTeX{}u sa obrázok konvertuje do formátu PDF, napr.
% pomocou programu \texttt{epstopdf}. Zvyčajne sa číslujú vzťahy, na
% ktoré sa v~texte odvolávame. Napríklad: vzťahy (\ref{r:1}) definujú
% Cauchyho začiatočnú úlohu.


% Original image reference commented out because the file doesn't exist
% \begin{figure}[ht!]
% \centering
% \includegraphics[width=0.7\textwidth]{tlmosc}
% \caption{Grafické zobrazenie riešenia rovnice \eqref{r:1}}\label{o:2}
% \end{figure}

% Place your own figure here when ready, for example:
% \begin{figure}[ht!]
% \centering
% \includegraphics[width=0.7\textwidth]{your_image_filename}
% \caption{Your caption here}\label{o:2}
% \end{figure}



% \subsection{Podkapitola}
% Podkapitoly záverečnej práce majú za úlohu členenie textu záverečnej
% práce s~cieľom, čo najväčšej prehľadnosti. Kapitol môže byť viacero
% a~v~ich názvoch sa používa desatinné číslovanie.