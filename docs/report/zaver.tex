% \section{Z\'aver (zhodnotenie rie\v{s}enia)}
% 
% Táto časť\/ záverečnej práce je povinná. Autor uvedie zhodnotenie
% riešenia. Uvedie výhody, nevýhody riešenia,  použitie výsledkov, ďalšie
% možnosti a~pod., prípadne načrtne iný spôsob riešenia úloh, resp.
% uvedie, prečo postupoval uvedeným spôsobom.
%  
% 
% Na strane \pageref{o:nlp} Vám ukážem ako sa používajú citácie.
% Ukážka:
% 
% \cite{Sumichrast2013}
% 
% \citep{Sumichrast2013}
% 
% \citealt{Sumichrast2013}
% 
% \citeyear{Sumichrast2013}
% 
% \citeauthor{Sumichrast2013}


% \begin{figure}
% \centering
% \includegraphics[width=.95\textwidth,angle=0]{natlib.PNG}
% \caption{Ukážka použitia \texttt{natlib package}}\label{o:nlp}
% \end{figure}

\section{Záver}

Cieľom tejto práce bolo navrhnúť, implementovať a vyhodnotiť optimalizačný rámec pre model GRU, ktorý umožňuje efektívnu predikciu na základe časových radov. V priebehu riešenia sme aplikovali tri rôzne prístupy k optimalizácii hyperparametrov: Random Search, Bayesian Search a Grid Search. Finálna verzia modelu bola trénovaná s využitím najlepšej kombinácie hyperparametrov získaných z Grid Search. Následná analýza preukázala vysokú generalizačnú schopnosť a stabilitu modelu.

\subsection{Zhrnutie dosiahnutých výsledkov}

\begin{itemize}
\item Bol navrhnutý a implementovaný univerzálny rámec pre hľadanie optimálnych hyperparametrov GRU modelov.
\item Vyvinutý model dosiahol výborné výsledky na reálnej dátovej množine (požičky bicyklov v Londýne).
\item Získané výsledky boli dôkladne analyzované pomocou metrických ukazovateľov (RMSE, MAE, R\textsuperscript{2}, MAPE, SMAPE, atď.).
\item Bola overená účinnosť kombinovaného prístupu k optimalizácii – Random a Bayesian Search na zúženie priestoru a Grid Search pre detailné ladenie.
\item Najlepší model dosiahol hodnotu R\textsuperscript{2} = 0.83122 na validačnej množine, čo predstavuje vysoko presný výsledok vzhľadom na komplexnosť úlohy.
\end{itemize}

\subsection{Hodnotenie kvality riešenia}

Model vykazuje:

\begin{itemize}
\item Rýchlu konvergenciu počas trénovania bez známok výrazného overfittingu.
\item Stabilitu počas celého tréningového procesu.
\item Výborný súlad medzi trénovacou a validačnou množinou.
\item Robustnosť voči výkyvom v dátach, ako ukazujú metriky PEAK ERROR a SMAPE.
\end{itemize}

Zvolený prístup sa ukázal ako efektívny a prakticky realizovateľný aj pri obmedzených výpočtových zdrojoch, najmä vďaka kombinácii adaptívnych a systematických metód optimalizácie.

\subsection{Odporúčania a návrhy na zlepšenie}

\begin{itemize}
\item Rozšíriť množinu testovaných metód o ďalšie optimalizačné techniky ako napr. Tree-structured Parzen Estimator (TPE) alebo Hyperband.
\item Preskúmať vplyv iných architektúr (napr. LSTM) pre porovnanie s GRU.
\item Implementovať možnosti automatickej predikcie na produkčné dáta v reálnom čase.
\end{itemize}

\subsection{Záverečné zhodnotenie}

Zvolený prístup bol v praxi úspešný a viedol k vytvoreniu predikčného systému, ktorý je nielen presný, ale aj stabilný a efektívne navrhnutý. Model GRU trénovaný na optimalizovanej sade hyperparametrov dosiahol veľmi dobré výsledky a preukázal potenciál využitia pri predikcii časových radov v rôznych doménach. Celý vývojový cyklus — od analýzy problému cez výber metód, implementáciu, optimalizáciu až po vizualizáciu a vyhodnotenie — bol realizovaný systematicky a metodicky správne.
