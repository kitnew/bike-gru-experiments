% !TeX encoding = UTF-8
% !TeX spellcheck = sk_SK
% !TeX root=tukedip.tex
\setcounter{page}{1}
\setcounter{equation}{0}
\setcounter{figure}{0}
\setcounter{table}{0}

\section*{Úvod}
\addcontentsline{toc}{section}{\numberline{}Úvod}
Význam strojového učenia neustále narastá, čo vedie k vzniku množstva rôznych modelov a architektúr schopných riešiť široké spektrum úloh, od predikcie časových radov po spracovanie prirodzeného jazyka. Medzi najpopulárnejšie modely patria rekurentné neurónové siete (RNN), ktoré sú vhodné na spracovanie sekvenčných dát vďaka schopnosti pamätať si predchádzajúce vstupy. Jednou z efektívnych a rozšírených architektúr RNN je Gated Recurrent Unit (GRU).

Táto práca sa zameriava na problematiku optimalizácie hyperparametrov GRU modelov. Správny výber a nastavenie hyperparametrov má zásadný vplyv na výkonnosť a presnosť modelu. Implementácia musí byť dostatočne univerzálna na použitie s rôznymi dátovými množinami bez nutnosti zásadných úprav kódu. V rámci zadania bolo nutné všetky metriky hodnotenia implementovať manuálne, čo umožňuje hlbšie pochopenie a kontrolu nad celým procesom učenia.